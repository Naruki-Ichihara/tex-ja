\documentclass[10.5ptj, a4paper, uplatex, dvipdfmx]{jsarticle}
\usepackage{lectureNote}

\title{直列回路の基礎実験} 
\author{市原稔紀 \thanks{E-mail: ichihara.naruki@nihon-u.ac.jp}}
\date{\today}
\begin{document}
\maketitle
\begin{abstract}
    本講義資料では,基本的な受動素子を直列に接続した回路システムの周波数応答および過渡応答の理論的な取り扱いを解説する.
\end{abstract}
\section{周波数応答}
\subsection{正弦波交流}
正弦波交流電圧$e$および正弦波交流電流$i$の瞬時値はそれぞれ次のように表すことができる.
\begin{equation}
    e = E_m\sin(\omega t),
\end{equation}
\begin{equation}
    i = I_m\sin(\omega t + \phi).
\end{equation}
ここで$E_m, I_m$はそれぞれ最大電圧および最大電流,$\phi$は電圧を基準とした位相,$t$は時間である.また$\omega$は角周波数であり,周期$T$および周波数$f$と次のような関係にある.
\begin{equation}
    \omega = \frac{2\pi}{T}=2\pi f
\end{equation}
\end{document}